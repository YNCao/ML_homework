%%%%%%%%%%%%%%%%%%%%%%%%%%%%%%%%%%%%%%%%%%%%%%%%%%%%%%%%%%%%%%%%
%
%  Template for homework of Introduction to Machine Learning.
%
%  Fill in your name, lecture number, lecture date and body
%  of homework as indicated below.
%
%%%%%%%%%%%%%%%%%%%%%%%%%%%%%%%%%%%%%%%%%%%%%%%%%%%%%%%%%%%%%%%%


\documentclass[11pt,letter,notitlepage]{article}
%Mise en page
\usepackage[left=2cm, right=2cm, lines=45, top=0.8in, bottom=0.7in]{geometry}
\usepackage{fancyhdr}
\usepackage{fancybox}
\usepackage{graphicx}
\usepackage{pdfpages} 
\usepackage{enumitem}
\renewcommand{\headrulewidth}{1.5pt}
\renewcommand{\footrulewidth}{1.5pt}
\pagestyle{fancy}
\newcommand\Loadedframemethod{TikZ}
\usepackage[framemethod=\Loadedframemethod]{mdframed}

\usepackage{amssymb,amsmath}
\usepackage{amsthm}
\usepackage{thmtools}

\setlength{\topmargin}{0pt}
\setlength{\textheight}{9in}
\setlength{\headheight}{0pt}

\setlength{\oddsidemargin}{0.25in}
\setlength{\textwidth}{6in}

%%%%%%%%%%%%%%%%%%%%%%%%
%% Define the Exercise environment %%
%%%%%%%%%%%%%%%%%%%%%%%%
\mdtheorem[
topline=false,
rightline=false,
leftline=false,
bottomline=false,
leftmargin=-10,
rightmargin=-10
]{exercise}{\textbf{Exercise}}
%%%%%%%%%%%%%%%%%%%%%%%
%% End of the Exercise environment %%
%%%%%%%%%%%%%%%%%%%%%%%

%%%%%%%%%%%%%%%%%%%%%%%
%% Define the Solution Environment %%
%%%%%%%%%%%%%%%%%%%%%%%
\declaretheoremstyle
[
spaceabove=0pt, 
spacebelow=0pt, 
headfont=\normalfont\bfseries,
notefont=\mdseries, 
notebraces={(}{)}, 
headpunct={:\quad}, 
headindent={},
postheadspace={ }, 
postheadspace=4pt, 
bodyfont=\normalfont, 
qed=$\blacksquare$,
preheadhook={\begin{mdframed}[style=myframedstyle]},
	postfoothook=\end{mdframed},
]{mystyle}

\declaretheorem[style=mystyle,title=Solution,numbered=no]{solution}
\mdfdefinestyle{myframedstyle}{%
	topline=false,
	rightline=false,
	leftline=false,
	bottomline=false,
	skipabove=-6ex,
	leftmargin=-10,
	rightmargin=-10}
%%%%%%%%%%%%%%%%%%%%%%%
%% End of the Solution environment %%
%%%%%%%%%%%%%%%%%%%%%%%


% Definition environment
\theoremstyle{definition}
\newtheorem{definition}{Definition}


%% Homework info.
\newcommand{\posted}{\text{Mar. 12, 2020}}       			%%% FILL IN POST DATE HERE
\newcommand{\due}{\text{Mar. 19, 2020}} 			%%% FILL IN Due DATE HERE
\newcommand{\hwno}{\text{2}} 		           			%%% FILL IN LECTURE NUMBER HERE


%%%%%%%%%%%%%%%%%%%%
%% Put your information here %%
%%%%%%%%%%%%%%%%%%%
\newcommand{\name}{\text{Yunning Cao}}  	          			%%% FILL IN YOUR NAME HERE
\newcommand{\id}{\text{PB16021370}}		       			%%% FILL IN YOUR ID HERE
%%%%%%%%%%%%%%%%%%%%
%% End of the student's info %%
%%%%%%%%%%%%%%%%%%%


\newcommand{\proj}[2]{\textbf{P}_{#2} (#1)}
\newcommand{\lspan}[1]{\textbf{span}  (#1)  }
\newcommand{\rank}[1]{ \textbf{rank}  (#1)  }
\newcommand{\dom}{ \textbf{dom}  }
\newcommand{\RNum}[1]{\uppercase\expandafter{\romannumeral #1\relax}}
% \def\cl{\mathbf{cl}}
% \def\int{\mathbf{int}}
% \def\conv{\mathbf{conv}}
\DeclareMathOperator*{\intp}{\bf int\,}
\DeclareMathOperator*{\cl}{\bf cl\,}
\DeclareMathOperator*{\bd}{\bf bd\,}
\DeclareMathOperator*{\conv}{\bf conv\,}
\DeclareMathOperator*{\epi}{\bf epi\,}


\lhead{
	\textbf{\name}
}
\rhead{
	\textbf{\id}
}
\chead{\textbf{
		Homework \hwno
	}}
	
	
	\begin{document}
		\vspace*{-4\baselineskip}
		\thispagestyle{empty}
		
		
		\begin{center}
			{\bf\large Introduction to Machine Learning}\\
			{Fall 2019}\\
			University of Science and Technology of China
		\end{center}
		
		\noindent
		Lecturer: Jie Wang  			 %%% FILL IN LECTURER HERE
		\hfill
		Homework \hwno             			
		\\
		Posted: \posted
		\hfill
		Due: \due
		\\
		Name: \name             			
		\hfill
		ID: \id						
		\hfill
		
		\noindent
		\rule{\textwidth}{2pt}
		
		\medskip
		
		
		
		
		
		%%%%%%%%%%%%%%%%%%%%%%%%%%%%%%%%%%%%%%%%%%%%%%%%%%%%%%%%%%%%%%%%
		%% BODY OF HOMEWORK GOES HERE
		%%%%%%%%%%%%%%%%%%%%%%%%%%%%%%%%%%%%%%%%%%%%%%%%%%%%%%%%%%%%%%%%
		
		\textbf{Notice, }to get the full credits, please present your solutions step by step.
		
		\begin{exercise}[Limit and Limit Points \textnormal{10pts}]
			\begin{enumerate}
				\item Show that $\{\mathbf{x}_n\}$ in $\mathbb{R}^n$ converges to $\mathbf{x}\in \mathbb{R}^n$ if and only if $\{\mathbf{x}_n\}$ is bounded and has a unique limit point $\mathbf{x}$.
				\item (\textbf{Limit Points of a Set}). Let $C$ be a subset of $\mathbb{R}^n$. A point $\mathbf{x}\in \mathbb{R}^n$ is called a limit point of $C$ if there is a sequence $\{\mathbf{x}_n\}$ in $C$ such that $\mathbf{x}_n\to \mathbf{x}$ and $\mathbf{x}_n \not=\mathbf{x}$ for all positive integers $n$. If $\mathbf{x}\in C$ and $\mathbf{x}$ is not a limit point of $C$, then $\mathbf{x}$ is called an isolated point of $C$. Let $C^\prime$ be the set of limit points of the set $C$. Please show the following statements.
				\begin{enumerate}
					\item If $C = (0,1)\cup\{2\}\subset \mathbb{R}$, then $C^\prime =[0,1]$ and $x=2$ is an isolated point of $C$.
					\item  The set $C^\prime$ is closed.
					\item  The closure of $C$ is the union of $C^\prime$ and $C$; that is $\cl C=C^\prime \cup C$. Moreover, $C^\prime \subset C$ if and only if $C$ is closed.
				\end{enumerate}
			\end{enumerate}
		\end{exercise}
		\begin{solution}
            \textbf{Exercise 1}
            \begin{enumerate}
                \item 
               Proof:
               \begin{itemize}
                    \item 
                    \begin{align*}
                        &\forall \varepsilon >0, ~\exists N, ~s.t. ~\Vert \mathbf{x}_k-\mathbf{x} \Vert <\varepsilon, ~\forall k>N\\
                        \Rightarrow&\{\mathbf{x}_n\} \text{ is bounded and it has a unique limit point }\mathbf{x}
                    \end{align*}
                    
                    \item
                    \begin{align*}
                        &\forall \mathbf{x}_k\in \{\mathbf{x}_n\},~\exists M>0,~s.t.~\Vert \mathbf{x}_k\Vert\leqslant M.\\
                        \Rightarrow&
                    \end{align*}
               \end{itemize}
               
               \item
               \begin{enumerate}
                    \item
                    According to the density of real numbers, it is clear: $\forall x \in [0,1],~\exists $ subsequence ${x_n}$ that $x_n\rightarrow x$ and $x_n\neq x$. There exists no such subsequence for element \{2\}. \\
                    $\therefore C'=[0,1]$, \{2\} is isolated point of C.
                    
                    \item
                    $C'=[0,1],~\mathbf{cl}~C'=[0,1]$, i.e. the closure of C' is equal to C'.
                    
                    \item
                    \begin{itemize}
                        \item
                        it is obvious that : $C$ is closed $\Rightarrow~C=\mathbf{cl}C=C'\cup C~\Rightarrow C'\subset C$
                        \item
                        $C'\subset C\Rightarrow~\mathbf{cl}C=C'\cup C=C\Rightarrow C$ is closed.
                    \end{itemize}

               \end{enumerate}

            \end{enumerate}
		\end{solution}
		\newpage
		
		\begin{exercise}[Open and Closed Sets \textnormal{10pts}]
			The norm ball $\{\mathbf{y} \in \mathbb{R}^n:\|\mathbf{y}-\mathbf{x}\|_2<r, \mathbf{x}\in \mathbb{R}^n\}$ is denoted by $B_r(\mathbf{x})$.
			\begin{enumerate}
				\item Given a set $C \subset \mathbb{R}^n$, please show the following are equivalent.
				\begin{enumerate}
					\item The set $C$ is closed; that is $\cl C=C$. 
					\item The complement of $C$ is open.
					\item If $B_{\epsilon}(\mathbf{x})\cap C \not=\emptyset$ for every $\epsilon>0$, then $\mathbf{x}\in C$.
				\end{enumerate}
				\item Given $A\subset\mathbb{R}^n$, a set $C\subset A$ is called open in $A$ if $$C=\{\mathbf{x}\in C: B_{\epsilon}(\mathbf{x})\cap A \subset C\,\text{for some}\, \epsilon>0\}.$$
				A set $C$ is said to be closed in $A$ if $A\setminus C$ is open in $A$. 
				\begin{enumerate}
					\item Let $B= [0,1] \cup \{2\}$.  Please show that $[0,1]$ is not an open set in $\mathbb{R}$, while it is both open and closed in $B$.
					\item Please show that a set $C \subset A$ is open in $A$ if and only if $C=A\cap U$, where $U$ is open in $\mathbb{R}^n$.
				\end{enumerate}
				
			\end{enumerate}
		\end{exercise}
		
		\begin{solution}
		\textbf{Exercise 2}
			\begin{enumerate}
                \item
                According to definition,(a)$\Leftrightarrow$(b)\\\\
                the complement of $C\text{ is open}\\
                \Leftrightarrow \forall \mathbf{x}\in C^c,\mathbf{x}$ is interior point\\
                $\Leftrightarrow $if $B_{\epsilon}(\mathbf{x})\cap C \not=\emptyset$ for every $\epsilon>0$, then $\mathbf{x}\notin C^c$ i.e. $\mathbf{x}\in C$\\
                i.e. (b)$\Leftrightarrow$(c)
                
                \item
                \begin{enumerate}
                    \item
                    $\forall \varepsilon>0,~ B_{\varepsilon}(1)\cap \mathbb{R}=B_{\varepsilon}(1)$ is not in [0,1] $\Rightarrow$ [0,1] is not an open set in $\mathbb{R}$.\\\\
                    $\forall x \in [0,1],~\exists \varepsilon>0,~ B_{\varepsilon}(x)\cap B$ is in [0,1] $\Rightarrow$ [0,1] is an open set in $B$.\\
                    $B\setminus[0,1]=\{2\}$ is open $\Rightarrow$ [0,1] is an closed set in $B$.
                    
                    \item
                    $C=A\cap U \\
                    \Leftrightarrow \forall\mathbf{x}\in C,\mathbf{x} \in A $ and $\exists \varepsilon>0 , ~s.t.~ B_{\varepsilon}(\mathbf{x})\in U$\\
                    $\Leftrightarrow\forall \mathbf{x}\in C,~\exists \varepsilon>0 , ~s.t.~ B_{\varepsilon}(\mathbf{x})\in U$, $B_{\varepsilon}(\mathbf{x})\cap A\in U\cap A=C$\\
                    $\Leftrightarrow C$ is open in $A$.
                \end{enumerate}

			\end{enumerate}

		\end{solution}
		\newpage
		
		
		
		\begin{exercise}[Bolzano-Weierstrass Theorem \textnormal{10pts}]
			\textbf{The Least Upper Bound Axiom}
			
			
			\emph{Any nonempty set of real numbers with an upper bound has a least upper bound. That is, $\sup  C$ always exists for a nonempty bounded above set $C \subset \mathbb{R}$.}
			
			Please show the following statements from \textbf{the least upper bound axiom}.
			\begin{enumerate}
				\item Let $C$ be a nonempty subset of $\mathbb{R}$ that is bounded above. Prove that $u = \sup C$ if and only if $u$ is an upper bound of $C$ and
				\begin{align*}
					\forall\,\epsilon>0,\exists\,a \in C\,\text{ such that }\,a>u-\epsilon.
				\end{align*}
				\item  Every bounded sequence in $\mathbb{R}$ has at least one limit point.
				\item Every bounded sequence in $\mathbb{R}^n$ has at least one limit point.
			\end{enumerate}
		\end{exercise}
		
		\begin{solution}
		\textbf{Exercise 3}
			\begin{enumerate}
                \item
                $u=\sup C$\\
                $\Leftrightarrow$\\
                $u$ is an upper bound of C,\\
                if $u'$ is any upper bound of C,$u<u'$.\\
                $\Leftrightarrow$\\
                
                \item
                Let sequence ${x_n}\in\mathbb{R}$ is bounded, $\forall n, x_n\in[a,b]$.\\
                find $c\in [a,b]$ that set [a,b] into 2 equal subsets. There is at least one subset that contains a infinite subsequence of ${x_n}$, let the subset be $[a_k,b_k]$. Repeat the process and $|b_k-a_k|\rightarrow0$ we will finally find a limit point $x$ of ${x_n}$.
                
                \item
                Let sequence ${\mathbf{x}_n}\in\mathbb{R}^n$ is bounded, $\forall n, \Vert\mathbf{x}_n\Vert\in[a,b]$.\\
                find $c\in [a,b]$ that set [a,b] into 2 equal subsets. There is at least one subset that contains a infinite subsequence of ${\mathbf{x}_n}$, let the subset be $[a_k,b_k]$. Repeat the process and $|b_k-a_k|\rightarrow0$ we will finally find a limit point $\mathbf{x}$ of ${\mathbf{x}_n}$.
            \end{enumerate}

		\end{solution}
		
		
		\newpage
		
		
		
		\begin{exercise}[Extreme Value Theorem \textnormal{15pts}]
			\begin{enumerate}
				\item Show that a set $C \subset \mathbb{R}^n$ is compact if and only if every sequence in $C$ has a subsequence that converges to a point in $C$.
				\item Let $C$ be a compact subset of $\mathbb{R}^n$ and $f:C \rightarrow \mathbb{R}$ be continuous. Please show that there exist $\mathbf{a},\mathbf{b} \in C$ such that
				\begin{align*}
					f(\mathbf{a}) \leq f(\mathbf{x}) \leq f(\mathbf{b}),\,\forall\,\mathbf{x}\in C.
				\end{align*}
				\item Let $f: \left[ a,b \right] \rightarrow\mathbb{R}$ be continuous. Show that the range of $f$ is a compact interval $\left[c,d \right]$ for some $c,d \in \mathbb{R}$.
			\end{enumerate}
		\end{exercise}
		\begin{solution}
			\textbf{Exercise 4}
			\begin{enumerate}
                \item
                
			\end{enumerate}

		\end{solution}
		
		
		\newpage
		
		\begin{exercise}[Convex Sets \textnormal{10pts}]
			Let $C \subset \mathbb{R}^n$ be a convex set. Please show the following statements.
			\begin{enumerate}
				\item The intersection $\cap_{i \in I}C_i$ of any collection $\{ C_i:i\in I \}$ of convex sets is convex.
				\item Both $\cl C $ and $\intp C $ are convex.
				\item The set$\{ \mathbf{y}\in\mathbb{R}^m:\mathbf{y}=\mathbf{Ax}+\mathbf{a},\mathbf{x}\in C \}$ is convex, where $\mathbf{A} \in \mathbb{R}^{m \times n}$ and $\mathbf{a} \in \mathbb{R}^m$.
				\item The set $\{ \mathbf{y}\in\mathbb{R}^m:\mathbf{x}=\mathbf{By}+\mathbf{b},\mathbf{x}\in C \}$ is convex, where $\mathbf{B} \in \mathbb{R}^{n \times m}$ and $\mathbf{b} \in \mathbb{R}^n$.
			\end{enumerate}
		\end{exercise}
		\begin{solution}
			
		\end{solution}
		
		\newpage
		
		\begin{exercise}[Carathéodory’s Lemma \textnormal{5pts}]
			Suppose that $S \subset \mathbb{R}^n$. Show that every element of $\conv S$ is a convex combination of at most $n + 1$ points of $S$.
		\end{exercise}
		\begin{solution}
			
		\end{solution}
		\newpage
		
		\begin{exercise}[Strictly Convex Functions \textnormal{10pts}]
			\begin{enumerate}
				\item Suppose that $f:\mathbb{R}^n \to \mathbb{R}$ is continuously differentiable. Please show that $f$ is strictly convex if and only if 
				\begin{align*}
					f(\mathbf{y})> f(\mathbf{x})+\langle\nabla f(\mathbf{x}),\mathbf{y}-\mathbf{x}\rangle, \forall\, \mathbf{x},\mathbf{y} \in\mathbb{R}^n, \mathbf{x}\not=\mathbf{y}.
				\end{align*}
				\item  Suppose that $f:\mathbb{R}^n \to \mathbb{R}$ is twice continuously differentiable. Please show that $f$ is strictly convex if 
				\begin{align*}
					\nabla^2 f(\mathbf{x}) \succ 0, \forall\, \mathbf{x} \in \mathbb{R}^n.
				\end{align*}
				Is the converse true? Please show your statement.
			\end{enumerate}
		\end{exercise}
		\begin{solution}
			
		\end{solution}
		\newpage
		
		\newpage
		
		\begin{exercise}[Strongly Convex Functions \textnormal{15pts}]
			\begin{enumerate}
				\item  Suppose that $f$ is continuously differentiable. Show that a continuously differentiable function $f$ is strongly convex with parameter $\mu>0$ if and only if 
				\begin{align*}
					f(\mathbf{y})\ge f(\mathbf{x})+\langle\nabla f(\mathbf{x}),\mathbf{y}-\mathbf{x}\rangle+\frac{\mu}{2}\|\mathbf{y}-\mathbf{x}\|_2^2, \forall\, \mathbf{x},\mathbf{y}\in\mathbb{R}^n.
				\end{align*}
				\item  Suppose that $f$ is twice continuously differentiable and strongly convex with parameter $\mu>0$. Please give an interpretation of $\mu$ in terms of the eigenvalues of $\nabla^2f(\mathbf{x})$.
				\item (\textbf{Lipschitz Continuity}). Suppose that $f:\mathbb{R}^n\rightarrow\mathbb{R}$ is twice continuously differentiable, and the gradient of $f$ is Lipschitz continuous, i.e.,
				\begin{align*}
					\|\nabla f(\mathbf{x})-\nabla f(\mathbf{y})\|_2\le L\|\mathbf{x}-\mathbf{y}\|_2, \forall\,\mathbf{x},\mathbf{y}\in\mathbb{R}^n,
				\end{align*}
				where $L>0$ is the Lipschitz constant. Please give an interpretation of $L$ in terms of the eigenvalues of $\nabla^2f(\mathbf{x})$.
			\end{enumerate}
		\end{exercise}
		\begin{solution}
			
		\end{solution}
		
		
		
		\newpage
		
		
		
		
		
		\begin{exercise}[Epigraph \textnormal{10pts}]
			\begin{enumerate}
				\item Given a real-valued function $f:\mathbb{R}^n \to \left( - \infty,+\infty \right]$. Show that the following are equivalent:
				\begin{enumerate}
					\item The epigraph $\epi f$ is closed.
					\item The $\alpha$-level set $C_{\alpha}$ is closed for any value of $\alpha$.
				\end{enumerate}
				\item Let the function $f:\mathbb{R}^n \times \mathbb{S}^n_{++} \rightarrow \mathbb{R}$ be defined as
				\begin{align*}
					f(\mathbf{x},\mathbf{Y}) = \mathbf{x}^{\top}\mathbf{Y}^{-1}\mathbf{x},
				\end{align*}
				where $\mathbb{S}^n_{++}$ denotes the set of symmetric positive-definte $n \times n$ matrices. Please show that $f$ is convex.
			\end{enumerate}
		\end{exercise}
		\begin{solution}
			
		\end{solution}
		\newpage
		
		
		\begin{exercise}[ Operations that Preserve Convexity \textnormal{25pts}]
			\begin{enumerate}
				\item Let $f:\mathbb{R}^m \rightarrow \left( -\infty,+\infty \right]$ be a given convex function, let $\mathbf{A}\in \mathbb{R}^{m \times n}$ and $\mathbf{b} \in \mathbb{R}^m$, and let
				\begin{align*}
					F(\mathbf{x}) = f(\mathbf{Ax+b}),\,\mathbf{x}\in\mathbb{R}^n
				\end{align*}
				Show that $F$ is convex.
				\item Let $f_i:\mathbb{R}^n \rightarrow \left(-\infty,+\infty \right]$, $i=1,\dots,m$ be given convex functions. Show that
				\begin{align*}
					F(\mathbf{x}) = \sum_{i=1}^m w_if_i(\mathbf{x})
				\end{align*}
				is convex, where $w_i \geq 0,\,i=1,\dots,m$.
				\item \begin{enumerate}
					\item Let $f_i:\mathbb{R}^n \rightarrow \left(-\infty,+\infty \right]$ be given convex functions for $i \in I$, where $I$ is an arbitrary index set. Show that
					\begin{align*}
						F(\mathbf{x}) = \sup_{i\in I}f_i(\mathbf{x})
					\end{align*}
					is convex.
					\item Consider the function $f(\mathbf{X})=\lambda_{\max}(\mathbf{X})$, with $\dom\, f =\mathbb{S}^n$, where $\lambda_{\max}(\mathbf{X})$ is the largest eigenvalue of $\mathbf{X}$ and $\mathbb{S}^n$ is a set of $n\times n$ real symmetric matrices. Please show that $f$ is a convex function.
				\end{enumerate}
				\item Suppose that the training set is $\{(\mathbf{x}_i,y_i)\}_{i=1}^m$, where $\mathbf{x}_i\in\mathbb{R}^d$ is the $i^{th}$ data instance and $y_i\in\mathbb{R}$ is the corresponding label.
				\begin{enumerate}
					\item Lasso refers to the regression problem as follows:
					\begin{align*}
						\min_{\boldsymbol{\beta}}\,\frac{1}{2m}\|\mathbf{X}\boldsymbol{\beta}-\mathbf{y}\|_2^2+\lambda\|\boldsymbol{\beta}\|_1,
					\end{align*}
					where $\mathbf{X}\in\mathbb{R}^{m\times d}$ with its $i^{th}$ row being $\mathbf{x}_i^{\top}$, $\boldsymbol{\beta} \in \mathbb{R}^d$, and $\lambda>0$ is the regularization parameter.
					\item Logistic regression refers to the problem as follows:
					\begin{align*}
						\min_{\mathbf{w},b}\,\frac{1}{m}\sum_{i=1}^m\log\left(1+\exp(-y_i(\langle \mathbf{w},\mathbf{x}_i\rangle-b))\right),
					\end{align*}
					where $y_i\in\{1,-1\}$, $\mathbf{w}\in\mathbb{R}^d$, and $b\in\mathbb{R}$.
				\end{enumerate}
				Show that the objective functions in the above problems are convex.
			\end{enumerate}
		\end{exercise}
		\begin{solution}
			
		\end{solution}
		
		%%%%%%%%%%%%%%%%%%%%%%%%%%%%%%%%%%%%%%%%%%%%%%%%%%%%%%%%%%%%%%%%
		
	\end{document}
